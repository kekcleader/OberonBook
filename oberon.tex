\documentclass[b5paper, 11pt, twoside, onecolumn, openany]{memoir}

%%% PACKAGES 

\usepackage[utf8]{inputenc}
\usepackage[T2A]{fontenc}
\usepackage[russian]{babel}
\usepackage[final]{microtype} % Less badboxes
\usepackage{bookmark}
\usepackage{listings}
\usepackage{soulutf8}

% \usepackage{kpfonts} %Font

\usepackage{amsmath,amssymb,mathtools} % Math

% \usepackage{tikz} % Figures
\usepackage{graphicx} % Include figures

%%% CODE LISTINGS
\lstset{
  language=Oberon-2,
  extendedchars=\true,
  basicstyle=\ttfamily\footnotesize,
  commentstyle=\color{red},
  escapeinside={\%*}{*\%},
  frame=bottomline,
  numbers=left,
  stepnumber=1,
  tabsize=2,
  showstringspaces=false,
  keepspaces=true,
  breaklines=true,
  breakatwhitespace=false,
  columns=fullflexible
}
\lstdefinestyle{Oberon-2}{
  keywordstyle=\color{red},
  commentstyle=\itshape\color{purple!40!black}
}
\newcommand{\kod}{\lstinline}

%%% PAGE LAYOUT 

\setlrmarginsandblock{0.15\paperwidth}{*}{1} % Left and right margin
\setulmarginsandblock{0.2\paperwidth}{*}{1}  % Upper and lower margin
\checkandfixthelayout

%%% SECTIONAL DIVISIONS

\maxsecnumdepth{section} % Subsections (and higher) are numbered
\setsecnumdepth{section}

\makeatletter %
\makechapterstyle{standard}{
  \setlength{\beforechapskip}{0\baselineskip}
  \setlength{\midchapskip}{1\baselineskip}
  \setlength{\afterchapskip}{2\baselineskip}
  \renewcommand{\chapterheadstart}{\vspace*{\beforechapskip}}
  \renewcommand{\chapnamefont}{\centering\normalfont\Large}
  \renewcommand{\printchaptername}{\chapnamefont \@chapapp}
  \renewcommand{\chapternamenum}{\space}
  \renewcommand{\chapnumfont}{\normalfont\Large}
  \renewcommand{\printchapternum}{\chapnumfont \thechapter}
  \renewcommand{\afterchapternum}{\par\nobreak\vskip \midchapskip}
  \renewcommand{\printchapternonum}{\vspace*{\midchapskip}\vspace*{5mm}}
  \renewcommand{\chaptitlefont}{\centering\bfseries\LARGE}
  \renewcommand{\printchaptertitle}[1]{\chaptitlefont ##1}
  \renewcommand{\afterchaptertitle}{\par\nobreak\vskip \afterchapskip}

  % Part titles
  \renewcommand{\beforepartskip}{}
  \renewcommand{\afterpartskip}{\bigskip}

  \renewcommand{\clearforchapter}{}
}
\makeatother

\chapterstyle{standard}

\setsecheadstyle{\normalfont\large\bfseries}
\setsubsecheadstyle{\normalfont\normalsize\bfseries}
\setparaheadstyle{\normalfont\normalsize\bfseries}
\setparaindent{0pt}\setafterparaskip{0pt}

%%% FLOATS AND CAPTIONS
%%%------------------------------------------------------------------------

\makeatletter % You do not need to write [htpb] all the time
\renewcommand\fps@figure{htbp}
\renewcommand\fps@table{htbp}
\makeatother

\captiondelim{\space } % A space between caption name and text
\captionnamefont{\small\bfseries} % Font of the caption name
\captiontitlefont{\small\normalfont} % Font of the caption text

\changecaptionwidth          % Change the width of the caption
\captionwidth{1\textwidth} %

%%% ABSTRACT
%%%------------------------------------------------------------------------

\renewcommand{\abstractnamefont}{\normalfont\small\bfseries} % Font of abstract title
\setlength{\absleftindent}{0.1\textwidth} % Width of abstract
\setlength{\absrightindent}{\absleftindent}

%%% HEADER AND FOOTER 
%%%------------------------------------------------------------------------

\makepagestyle{standard} % Make standard pagestyle

\makeatletter                 % Define standard pagestyle
\makeevenfoot{standard}{}{}{} %
\makeoddfoot{standard}{}{}{}  %
\makeevenhead{standard}{\bfseries\thepage\normalfont\qquad\small\leftmark}{}{}
\makeoddhead{standard}{}{}{\small\rightmark\qquad\bfseries\thepage}
% \makeheadrule{standard}{\textwidth}{\normalrulethickness}
\makeatother                  %

\makeatletter
\makepsmarks{standard}{
\createmark{chapter}{both}{nonumber}{\@chapapp\ }{ \quad }
\createmark{chapter}{right}{nonumber}{}{ \quad }
\createmark{section}{right}{nonumber}{}{ \quad }
\createmark{subsection}{right}{nonumber}{}{ \quad }
\createplainmark{toc}{both}{\contentsname}
\createplainmark{lof}{both}{\listfigurename}
\createplainmark{lot}{both}{\listtablename}
\createplainmark{bib}{both}{\bibname}
\createplainmark{index}{both}{\indexname}
\createplainmark{glossary}{both}{\glossaryname}
}
\makeatother                               %

\makepagestyle{chap} % Make new chapter pagestyle

\makeatletter
\makeevenfoot{chap}{}{}{} % Define new chapter pagestyle
\makeoddfoot{chap}{}{}{}  %
\makeevenhead{chap}{}{}{}   %
\makeoddhead{chap}{}{}{}    %
% \makeheadrule{chap}{\textwidth}{\normalrulethickness}
\makeatother

\nouppercaseheads
\pagestyle{standard}               % Choosing pagestyle and chapter pagestyle
\aliaspagestyle{chapter}{chap} %

%%% NEW COMMANDS
%%%------------------------------------------------------------------------

\maxtocdepth{subsection} % ToC depth
\settocdepth{subsection}

\AtEndDocument{\addtocontents{toc}{\par}} % Add a \par to the end of the TOC

\usepackage{hyperref}   % Internal hyperlinks
\hypersetup{
  pdfborder={0 0 0},      % No borders around internal hyperlinks
  pdfauthor={ФРА} % author
}
\usepackage{memhfixc}   %

\author{Г.~В.~Ф.~Гегель}
\title{Наука логики}
\date{}

\renewcommand{\partnumberline}[1]{} % Remove part number in ToC
\renewcommand{\cftpartdotsep}{\cftdotsep} % Part dots in ToC
\renewcommand{\chapternumberline}[1]{} % Remove chapter number in ToC
\renewcommand{\cftchapterdotsep}{\cftdotsep} % Chapter dots in ToC

% Remove section numbers in ToC
\let\oldcftsf\cftsectionfont% save definition of \cftsectionfont
\let\oldcftspn\cftsectionafterpnum% and of \cftsectionafterpnum
\renewcommand*{\cftsectionfont}{%
\let\oldnl\numberline% save definition of \numberline
\renewcommand*{\numberline}[1]{}% change it
\oldcftsf} % use original \cftsectionfont
\renewcommand*{\cftsectionafterpnum}{%
\let\numberline\oldnl% % restore orginal \numberline
\oldcftspn} % use original \cftsectionafterpnum

\begin{document}

\mainmatter

\begin{center}
  {А. И. Ефимов} \\
  {П. В. Александрян} \\
  {Д. В. Высоцкий} \\
  {В. И. Лукьянович} \\
  {Д. И. Воронович} \\
  \vspace{120pt}
  \textbf{\Huge ПОСТРОЕНИЕ} \\
  ~ \\
  \textbf{\Huge КОМПИЛЯТОРОВ} \\
  \vspace{180pt}
  {ИЗДАТЕЛЬСТВО} \\
  {ТЕХНИКО-ТЕОРЕТИЧЕСКОЙ ЛИТЕРАТУРЫ} \\
  {КОМАНДЫ РАЗРАБОТЧИКОВ} \\
  {КОМПИЛЯТОРА ОБЕРОНА} \\
  ~ \\
  {Р И Г А ~ ~ ~  2 0 1 9}
\end{center}

\clearpage

\part{Введение в устройство компилятора}

\chapter{Общее подразделение компилятора}
Компилятор преобразует исходный код программы в машинный код
электронно-вычислительной машины и состоит из следующих модулей:

\begin{itemize}
  \item драйвер текста,
  \item лексический анализатор,
  \item синтаксический анализатор,
  \item символьная таблица,
  \item кодогенератор.
\end{itemize}

{\bfseries Драйвер текста} разбивает исходный код программы
на~отдельные литеры и~передаёт~их по~одной лексическому
анализатору\footnote{В~указанном взаимоотношении не драйвер
текста, а~лексический анализатор играет ведущую роль "--- он
вызывает процедуру драйвера текста (\kod{ReadCh}), которая
только тогда считывает очередную литеру.}.

{\bfseries Лексический анализатор} принимает литеры одну за другой
и собирает их в лексемы языка программирования. Примеры лексем:
\kod{MODULE}, \kod{BEGIN}, \kod{+}, \kod{:=}, \kod{"Мики-джаки!"},
\kod{47.813}, \kod{>=}, \kod{myVariable}.
Лексемы пердаются синтаксическому анализатору по одной, при этом сам
лексический анализатор ровным счётом ничего не понимает (если здесь
может идти речь о~понимании) в~том, чт\'{о}
из~этих лексем может получиться и~как они должны взаимодействовать
друг с~другом. Другими словами, на~данном этапе ничего не известно
о~{\em синтаксисе} языка, известно только то, какие {\em слова}
существуют в~этом языке.
Лексический анализатор представлен внешнему миру
(т.\,е.\,синтаксическому анализатору) процедурой \kod{Get}.

{\bfseries Синтаксический анализатор} собирает лексемы в синтаксические
конструкции в соответствии с описанием языка программирования,
такие как \kod{IF условие THEN действие END} и
\kod{обозначение := выражение}\footnote{В данных примерах слов\'{а}
<<условие>>, <<действие>> и~т.\,д. подразумевают под собой некоторые
другие синтаксические конструкции.}.

Синтаксический анализатор представляет собой главный модуль компилятора.
Именно он руководит всей его работой.

{\bfseries Символьная таблица} "--- это вспомогательный модуль, который
используется синтаксическим анализатором для хранения данных об
объявленных в программе переменных, процедурах и~т.\,д.\footnote{Типы,
константы, переменные и процедуры вместе называются объектами.}
Эти данные используются в процессе компиляции. Символьная таблица также
отвечает за импорт модулей в компилируемую программу (таких, например,
как \kod{In} и \kod{Out}).

{\bfseries Кодогенератор} содержит набор процедур, которые вызываются
синтаксическим анализатором в процессе разбора исходного кода и
помещают в выходной файл генерируемый машинный код.

\subsection*{Итак, общая схема такова:\ \protect
\footnote{Слова, указанные в~скобках означают передаваемые данные,
а~слова без скобок "--- хранители или обработчики данных.}}

\begin{center}
\fbox{
\parbox{10cm}{
Файл с исходным кодом <<\kod{.Mod}>> $\rightarrow$
(Исходный код программы) $\rightarrow$
Драйвер текста $\rightarrow$
(Литеры) $\rightarrow$
Лексический анализатор $\rightarrow$
(Лексемы) $\rightarrow$
Синтаксический анализатор и символьная таблица $\rightarrow$
Кодогенератор $\rightarrow$
(Машинный код) $\rightarrow$
Файл с~машинным кодом}}
\end{center}

Как видно, данный процесс порождает файл с~машинным кодом.
В~простейшем случае (когда вся программа состоит из одного модуля),
на выходе получается исполнимый файл. В~ДОС и~Windows
такой файл имеет расширение
<<\kod{.exe}>>\footnote{Исполнимые файлы ДОС могут также иметь
расширение <<\kod{.com}>> "--- такой файл содержит чистый машинный код
без каких либо дополнительных данных. Он загружается в~ОЗУ (всегда
в~одно и~то~же место) и просто запускается.}, тогда
как в~других операционных системах он может вовсе не~иметь никакого
расширения.

В~более сложном случае, программа состоит из нескольких модулей, каждый
из которых компилируется отдельно от остальных, в~результате чего из
каждого \kod{Mod}-файла получается два файла:
<<\kod{.sym}>> и <<\kod{.o}>>.

Файл с расширением <<\kod{.o}>> называется \so{объектным файлом}.
Он содержит машинный код какого-то одного скомпилированного модуля.
<<\kod{.sym}>>-файл содержит данные об этом модуле, необходимые для того,
чтобы из всех <<\kod{.o}>>-файлов собрать исполнимый файл. Этот процесс
называется \so{компоновкой}.

\newpage


\chapter{Начинаем писать компилятор}

\section{Простейший драйвер текста}
\dots


\section{Простейший лексический анализатор}
\lstinputlisting{sources/100_lex1.Mod}



\backmatter

\clearpage
\tableofcontents*
\clearpage

\end{document}
