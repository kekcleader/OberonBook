Драйвер текста представляет собой процедуру \kod{ReadCh}.
Её задача "--- считывать литеры из потока входных данных
(исходного кода компилируемой программы) одну за другой.
Результат работы процедуры \kod{ReadCh} помещается в глобальную
переменную \kod{ch} типа \kod{CHAR}.

В простейшем случае драйвер текста может брать код программы
из переменной строкового типа.
Тогда ещё необходима переменная-бегунок по строке "--- \kod{i}.

\begin{lstlisting}
VAR s: ARRAY 64 OF CHAR;
  i: INTEGER;
\end{lstlisting}
